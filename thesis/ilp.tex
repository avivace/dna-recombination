\section{Definition} 
\textit{Linear programming} is a technique for the mathematical optimization of a linear objective function, subject to linear equality and linear inequality constraints.

Linear programs are problems that can be expressed in canonical form as:
\paragraph{}

${\displaystyle {\begin{aligned}
	&{\text{maximize}}&&\mathbf {c} ^{\mathrm {T} }\mathbf {x}
	\\&{\text{subject to}}&&A\mathbf {x} \leq \mathbf {b}
	\\&{\text{and}}&&\mathbf {x} \geq \mathbf {0}
	\\&&&{\text{(}}\mathbf {x} \in \mathbb {Z} ^{n}\text{)}
	\\&
	\end{aligned}}} $


If the variables are forcibly constrained to be integers, we call the program \textit{Integer} or \textit{Integer Linear}.

0-1 integer programming or binary integer programming (\textit{BIP}) is the special case of integer programming where variables are required to be 0 or 1 ($\mathbf {x} \in \mathbb \{0,1\} $).

In contrast to linear programming, which can be solved efficiently in the worst case, integer programming problems are in many practical situations (bounded variables) NP-hard. BIP are classified as NP-hard too ("0–1 integer programming" is one of the \textit{Karp's 21 NP-complete problems}).

\section{In Computational Biology}
At its inception, the focus of Computational Biology was on the development of efficient algorithms and data structures that were able to deal with the data being introduced in life science applications. In the last decade (?), the introduction of high throughput methods for biomedical data analysis and the rise of systems biology (?) made Statistical Learning approaches as a standard. This element and the fact that biological processes are usually reduced and studied as simulations (because the actual nature of them is still being investigated, as in the case of our problem) lead to the introduction of a lot new Optimization problems in the field.




\section{Examples}
\subsection{Problem 1}