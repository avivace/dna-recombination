\section{Definition} 
\textit{Linear programming} is a technique for the mathematical optimization of a linear objective function, subject to linear equality and linear inequality constraints.

Linear programs are problems that can be expressed in canonical form as:
\paragraph{}

${\displaystyle {\begin{aligned}
	&{\text{maximize}}&&\mathbf {c} ^{\mathrm {T} }\mathbf {x}
	\\&{\text{subject to}}&&A\mathbf {x} \leq \mathbf {b}
	\\&{\text{and}}&&\mathbf {x} \geq \mathbf {0}
	\\&&&{\text{(}}\mathbf {x} \in \mathbb {Z} ^{n}\text{)}
	\\&
	\end{aligned}}} $


If the variables are forcibly constrained to be integers, we call the program \textit{Integer} or \textit{Integer Linear}.

0-1 integer programming or binary integer programming (\textit{BIP}) is the special case of integer programming where variables are required to be 0 or 1 ($\mathbf {x} \in \mathbb \{0,1\} $).

In contrast to linear programming, which can be solved efficiently in the worst case, integer programming problems are in many practical situations (bounded variables) NP-hard. BIP are classified as NP-hard too ("0–1 integer programming" is one of the \textit{Karp's 21 NP-complete problems}).

\section{In Computational Biology}

\section{Examples}
\subsection{Problem 1}