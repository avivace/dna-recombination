\section{Definition} 
\textit{Linear programming} (ILP) is a technique for the mathematical optimization of a linear objective function, subject to linear equality and linear inequality constraints.

Linear programs are problems that can be expressed in canonical form as:
\paragraph{}

${\displaystyle {\begin{aligned}
	&{\text{maximize}}&&\mathbf {c} ^{\mathrm {T} }\mathbf {x} \qquad \text{(cost function)}
	\\&{\text{subject to}}&&A\mathbf {x} \leq \mathbf {b}
	\\&{\text{and}}&&\mathbf {x} \geq \mathbf {0}
	\\&&&{\text{(}}\mathbf {x} \in \mathbb {Z} ^{n}\text{)}
	\\&
	\end{aligned}}} $


If the variables are forcibly constrained to be integers, we call the program \textit{Integer} or \textit{Integer Linear} (ILP).

0-1 integer programming or binary integer programming (BIP) is the special case of integer programming where variables are required to be 0 or 1 ($\mathbf {x} \in \mathbb \{0,1\} $).

A particular case of integer linear programming is represented by Combinatorial Optimization (CO), that is the class of problems in which the feasible region is a subset of the vertices of the unit hypercube $F \subseteq \textbf{B}^{n} = \{0,1\}^{n} $, i.e., more simply, problems in which variables can only take value 0 or 1. Linear \{0,1\} (or binary) programming problems belong to this class \cite{ro-malucelli}.

In contrast to linear programming, which can be solved efficiently in the worst case, integer programming problems are in many practical situations (bounded variables) NP-hard and no general algorithm is known. Binary Integer Programming problems are classified as NP-hard too: "0–1 integer programming" is one of the \textit{Karp's 21 NP-complete problems}.

\section{Algorithms}
There are three main categories of algorithms for solving integer linear programming problems \cite{ilpalg}:
\begin{itemize}
	\item \textsc{Exact} algorithms that guarantee to find an optimal solution, but may take an exponential number of iterations. They include cutting-planes, branch-and-bound, and dynamic programming.
	\item \textsc{Heuristic} algorithms that provide a suboptimal solution, but without a guarantee on its quality. Although the running time is not guaranteed to be polynomial, empirical evidence suggests that some of these algorithms find a good solution fast.
	\item \textsc{Approximation} algorithms that provide in polynomial time a suboptimal solution together with a bound on the degree of sub-optimality.
\end{itemize}

\section{In Computational Biology}

At its inception, the focus of Computational Biology was on the development of efficient algorithms and data structures that were able to deal with the data being introduced in life science applications. Lately, the introduction of high throughput methods for biomedical data analysis and the rise of Systems Biology (the study of systems of biological components) made Statistical Learning approaches a standard \cite{ilpinb}.

Furthermore, new and accessible sequencing methods caused an exponential growth of the available genomic data.

This element and the fact that biological processes are usually reduced and studied as simulations (because the actual nature of them is still being investigated, as in the case of our problem) lead to the introduction of a lot new optimization problems in the field.

In most cases, these optimisazion problems are discrete ones: hence the approval of ILP-based approaches as a standard.

Some of the most successful Integer Programming approaches for computational biology problems are described in \cite{lancia2004}.

\subsection{Advantages}

There are a number of additional reasons why ILP should be taken into consideration, even when the problems seems to not require it or the advantage of introducing an ILP formulation isn't initally clear\cite{gusfieldilp}:

\begin{itemize}
	\item Commercial ILP \textit{solvers} are available (with academic licenses);
	\item The progress of those solvers has been spectacular: benchmark ILP problems can be solved \textit{200-bilion} times faster than twenty-years ago;
	\item Even for a problem where a worst-case efficient general algorithm might be possible, the time and effort needed to find it, implement it as a computer program, is typically much greater than the time and effort needed to formulate and implement an ILP solution to the problem.
	\item Some problems can be modeled in a much more efficient way with ILP.
	\item A new mathematical formulation for classic problems may be studied, allowing the original one to be attacked in new ways. New techniques and relaxations can be applied.
\end{itemize}

To give a real example, the deeply studied \textsc{Multiple Sequence Alignment} problem \cite{carrillo1988multiple} shows how many different approaches, versions and formulations can be theorised and exploited: tt was reformulated as an optimisation problem introducing the concept of \textit{trace} in \cite{Kececioglu1993}, given branch-and-cut algorithms in \cite{KECECIOGLU2000143} and relaxations, such as \cite{Althaus2007}, which proposes a branch-and-bound algorithm with a Lagrangian relaxation. 

Among others, \cite{NET:NET10022} reduce the multiple alignment problem to the minimum routing cost tree (MRCT) problem, i.e., finding a spanning tree in a complete weighted graph, which minimizes the sum of the distances between each pair of nodes. They propose a Branch-and-Price algorithm for the MRCT problem and then use it. \cite{DBLP:conf/stringology/JustV00} reduce multiple sequence alignment to a facility location problem. The reduction is then used to provide a Polynomial Time Approximation Scheme for a certain class of multiple alignment problems.


\subsection{A typical approach}
Usually, 

\section{Design of an ILP formulation}

\subsection{Idioms}

Here's how many logic expressions can be expressed as linear disequalities without side effects or uncovered cases \cite{gusfieldilp}.

Suppose \textit{L} is an integer linear function of binary variables with \textit{M} being its upper limit and \textit{b} a positive integer. Many of these idioms can be reduced if some or all variables are binary, strictly positive, or bounded.

\paragraph{If-Then}

$$ L \geq b \rightarrow z$$
Linearly:
$$ L - (M \times z) \leq  b - 1$$


\paragraph{Only-If}

$$ \text{$z = 1$ only if $L \geq b$}$$
Let \textit{s} be the smallest value that L can achieve and set $m = s - b$. Linearly:
$$ L + m \times z \geq m + b$$

These two idioms can be used as building blocks for many more:

\paragraph{NAND \\} 

Let $L_1$ and $L_2$ be linear functions whose variables are bounded, and $L_1 \geq b_1$ and $L_2 \geq b_2$ . We require that
at \textit{most} one of the two linear inequalities is satisfied.

$$z_1 + z_2 \leq 1 $$
Where $z_1 = 1$ if $L_1 \geq b_1$ and $z_2 = 1$ if $L_2 \geq b_2$. We use the \textit{If-Then} twice idiom to express these two conditions.

\paragraph{OR \\}
Here we require that at \textit{least} one of the two linear inequalities is satisfied.
$$z_1 + z_2 \geq 1 $$
Followed by two \textit{Only-If} idioms to express $z_1 = 1$ \textit{only} if $L_1 \geq b_1$ and $z_2 = 1$ \textit{only} if $L_2 \geq b_2$.


\paragraph{XOR \\}
If we want \textit{exactly} one of the inequalities to be satisfied:

$$z_1 + z_2 = 1$$

Again, followed by two \textit{Only-If} idioms to express $z_1 = 1$ \textit{only} if $L_1 \geq b_1$ and $z_2 = 1$ \textit{only} if $L_2 \geq b_2$ and two \textit{If-Then} (if and only if).

\paragraph{Implied Satisfaction \\}
To express
$$ L_1 \geq b_1 \rightarrow L_2 \geq b_2$$
We need an \textit{If-Then} idiom for the first equality, an \textit{Only-If} idiom for the second and

$$z_1 \leq z_2$$

\paragraph{Not-Equal \\}

Suppose $Z_1$ and $Z_2$ are linear functions of integer variables whose values are bounded. Then $Z_1 - Z_2$ and $Z_2 - Z_1$ are bounded to integer values, too. Then, we can express
$$Z_1 \neq Z_2$$
as
$$(Z_1 - Z_2 \geq 1) \text{ OR } (Z_2 - Z_1 \geq 1) $$

Using our \textit{OR} idiom previously explained: let $s_1$ the lower bound for $Z_1 - Z_2$ and $s_2$ the lower bound for $Z_2 - Z_1$.
Set  $m_1 = s_1 - 1$ and $m_2 = s_2 - 1$. The final inequalities will be

$$(Z_1 - Z_2) + m_1 \times l_1 \geq m_1 + 1$$
$$(Z_2 - Z_1) + m_2 \times l_2 \geq m_2 + 1$$
$$ l_1 + l_2 \geq 1 $$