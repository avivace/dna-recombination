The entire software and documentation produced during the Stage experience and the thesis drafting, including initial and discarded attempts are available in a public Git \href{https://github.com/avivace/dna-recombination}{repository} \cite{avivace_repo} on GitHub.

\section{Tools}
The sperimental work of the Stage experience was done on a GNU/Linux Debian \texttt{buster/sid} workstation, making large of use of the shell and other tools:

\texttt{Git}, a distributed version control system, helped to keep track of every progress in the documents and the software produced.

\texttt{Python} \cite{Rossum:1995:PRM:869369} and its \texttt{IDLE} was used to quickly implement and experiment the algorithms and procedures. It's also the interface to the Gurobi interactive shell. \texttt{Ruby} was considered too.

\texttt{Gurobi Optimizer} \cite{gurobi} is a commercial optimization solver. It provides a Python interface to formulate linears problem and solve them in Python software.

The \texttt{TeX} typesetting engine (in the \texttt{LaTex} macros environment, with some some extensions like \texttt{BiBTex} and the \texttt{pdflatex} compiler) were used to produce the documentation, the thesis and the slides.

\section{Reduced artificial instance}
\label{gen}

Working on the entire genomes sequences would be prohibitive for such approach, and many of the existent solutions make assumptions on the nature of the genomes, as we've seen.

We take into consideration a reduced instance, considering a shorter sequence and only the main events (Scrambling, Inversion, Overlapping, Deletion).

A Python script which \textit{procedually} generates an instance of the problem with given specific characteristics has been developed, it takes the following parameters to shape the desired instance:

\begin{itemize}
	\item MIC length
	\item MDS quantity
	\item Overlap size
	\item Inverse rate
\end{itemize}

The generated instance consists in:

\begin{itemize}
	\item A (randomised) MIC DNA sequence
	\item A rearrangement map containing positions, inversion flags and annotation for every MDS, Pointer in both MIC and MAC
	\item The resulting MAC sequence
\end{itemize}

Running \texttt{\$ python gen.py}, we get:
\begin{verbatim}
Generated instance:

MDS     Start   End
0       3       9
1       21      27
2       30      38
3       40      50
4       13      19

P       Start1  End1    Start2  End2
1       7       9       21      23
2       25      27      30      32
3       35      38      40      43
4       46      50      13      17

MIC     ---AAATAT----TGGAGG--ATCGGT---GTAGAATT--ATTTCGTGGA----------
               ^^    ^^^^    ^^  ^^   ^^   ^^^  ^^^   ^^^^
MAC     AAATATCGGTAGAATTTCGTGGAGG
            ^^  ^^   ^^^   ^^^^
\end{verbatim}
The parameters used were:

\begin{itemize}
	\item $60$ as MAC length
	\item A random value in the $2-5$ range as MDS quantity
	\item $0$ as Inverse rate
	\item $30\%$ as Overlap size
\end{itemize}

The pointer regions are marked in both MIC and MAC. IESs are masked for clarity. The complete annotation map is exposed through four produced objects: \texttt{MDS\_MAC}, \texttt{MDS\_Mic}, \texttt{Pointer\_MIC} and \texttt{Pointer\_MAC}, easily navigable to fetch any information about the simulated process, e.g., \texttt{Pointer\_MIC[1]["Start2"]} and \texttt{Pointer\_MIC[1]["End2"]} gives the position of the second occurrence of the first pointer section in the MIC. A standard JSON object, serialising this data, is produced, too.

\subsection{(Proposed) Rearrangement map format}
\label{rmap}
Here we show a simple rearrangement map format adopted in this work. A specification gives a coherent and reliable way interpret the events represented by the map.
A JSON schema of the format specification is proposed in \texttt{rmapSchema.json} and it's further documented on the example library that handles it (\texttt{rmap.py}).

The necessity to \textit{apply} them on genomes and produce simulations is described in \ref{correctness}.

The following events can be described:

\begin{itemize}
	\item Deletion.
	\item Scrambling. The array index represents the final ordering in the MAC. The scrambled order can be computed using the \texttt{start} and \texttt{end} positions.
	\item Overlapping of pointer sections. A pointer section is a sequence common to 2 MDSs. The pointer can appear inversed in one or both occurences.
	\item Inversion. An MDS can appear in the resulting genome as the Watson-Creek reverse complement version of the one in the MIC.
\end{itemize}

Here's how a rearrangement map in this format looks like, generated by \textit{gen.py}.

\begin{verbatim}
{
  "mac_length": 18,
  "mic_length": 60,
  "mds": [
    {
      "start": 27,
      "end": 32,
      "inverted": 0
    },
    {
      "start": 3,
      "end": 10,
      "inverted": 0
    },
    {
      "start": 14,
      "end": 24,
      "inverted": 0
    }
  ],
  "pointers": [
    {
      "start1": 30,
      "end1": 32,
      "start2": 3,
      "end2": 5
    },
    {
      "start1": 8,
      "end1": 10,
      "start2": 14,
      "end2": 16
    }
  ]
}
\end{verbatim}

This file can now be used by \texttt{map.py} which reproduces the events described by the map on a given genome.

\section{ILP}
\label{ilp_form}
\subsection{Variables definitions}

$*Eq(i,j,h,l) = \begin{cases} 0 \\ 1, & \mbox{if } \code{MIC[i:j]} = \code{MAC[h:l]} \end{cases}$ \\\\\\
$*cwc(i,j,h,l) = \begin{cases} 0 \\ 1, & \mbox{if \code{MIC[i:j]} is the reverse complement of \code{MAC[h:l]}} \end{cases}$ \\\\\\
$MDS_{MICstart}(i,j) = \begin{cases} 0 \\ 1, & \mbox{if MDS } i\mbox{ starts at position } j \mbox{ in the MIC} \end{cases}$ \\\\\\
$MDS_{MICend}(i,j) = \begin{cases} 0 \\ 1, & \mbox{if MDS } i\mbox{ ends at position } j \mbox{ in the MIC} \end{cases}$ \\\\\\
$MDS_{MACstart}(i,j) = \begin{cases} 0 \\ 1, & \mbox{if MDS } i\mbox{ starts at position } j \mbox{ in the MAC} \end{cases}$ \\\\\\
$MDS_{MACend}(i,j) = \begin{cases} 0 \\ 1, & \mbox{if MDS } i\mbox{ ends at position } j \mbox{ in the MAC} \end{cases}$ \\\\\\
$Inv(i) = \begin{cases} 0 \\ 1, & \mbox{if MDS } i\mbox{ is inverted in the MAC } \end{cases}$ \\\\\\
$P_{start}(i,j) = \begin{cases} 0 \\ 1, & \mbox{if } MDS_{MACstart}(i,j) = 1 \mbox{, Pointer } i \mbox{ starts at position } j \mbox{ in the MAC} \end{cases}$ \\\\\\
$P_{end}(i,j) = \begin{cases} 0 \\ 1, & \mbox{if } MDS_{MACend}(i-1,j) = 1 \mbox{, Pointer } i \mbox{ ends at position } j \mbox{ in the MAC} \end{cases}$ \\\\\\
$Cov_{MACPOINT}(i,j) = \begin{cases} 0 \\ 1, & \mbox{if Pointer i covers the position j in the MAC} \end{cases}$ \\\\\\
$Cov_{MIC}(i,j) = \begin{cases} 0 \\ 1, & \mbox{if MDS } i\mbox{ covers the position } j \mbox{ in the MIC} \end{cases}$ \\\\\\
$Cov_{MAC}(i,j) = \begin{cases} 0 \\ 1, & \mbox{if MDS } i\mbox{ covers the position } j \mbox{ in the MAC} \end{cases}$ \\\\\\
Variables marked with $*$ will be populated during the preprocessing phase. \\\\\\

$\mbox{Objective Function:} \qquad min \mathlarger{\sum_{i,j} MDS_{MACstart}(i,j)}$
\clearpage
\subsection{Constraints}

\paragraph{MDS integrity and validity} $ $ \\\\\\
MDSs must correspond to identical or reverse and complemented substrings of MIC and MAC.
\\\\\\
$(1) \qquad MDS_{MICstart}(i,a) + MDS_{MICend}(i,b) + MDS_{MACstart}(i,c) + MDS_{MACend}(i,d) + Inv(i) \leq 5 cwc(a,b,c,d) \label{eq:someequation} \qquad \forall i,a,b,c,d$ \\\\\\
$(2) \qquad MDS_{MICstart}(i,a) + MDS_{MICend}(i,b) + MDS_{MACstart}(i,c) + MDS_{MACend}(i,d) \leq 4 Eq(a,b,c,d) \qquad \forall i,a,b,c,d $ \\\\\\
Each MDS can start one time, both in the MAC and the MIC. \\\\\\
$(3) \qquad \mathlarger{\sum_{j} MDS_{MICstart}(i,j)\leq 1 \qquad \forall i}$ \\\\\\
$(3b) \qquad \mathlarger{\sum_{j} MDS_{MACstart}(i,j)\leq 1 \qquad \forall i}$ \\\\\\
If an MDS starts, it must end too. \\\\\\
$(4) \qquad \mathlarger{\sum_{j} MDS_{MICend}(i,j) = \sum_{j} MDS_{MICstart}(i,j) \qquad \forall i}$ \\\\\\
$(4b) \qquad \mathlarger{\sum_{j} MDS_{MACend}(i,j) = \sum_{j} MDS_{MACstart}(i,j) \qquad \forall i}$ \\\\\\
\clearpage
\paragraph{Coverage} $ $
\\\\\\
$(5) \qquad \mathlarger{\sum_{l\leq j} MDS_{MICstart}(i,l) + \sum_{l>j}MDS_{MICend}(i,l) - 2Cov_{MIC}(i,j) = 0} \qquad \forall i,j $ \\\\\\
$(6) \qquad \mathlarger{\sum_{l\leq j} MDS_{MACstart}(i,l) + \sum_{l> j}MDS_{MACend}(i,l) - 2Cov_{MAC}(i,j) = 0} \qquad \forall i,j $ \\
\paragraph{Pointer Regions} $ $
\\\\\\
Each Pointer starts when the correspondent MDS does. \\\\\\
$(7) \qquad P_{start}(i,j) = MDS_{MACStart}(i,j) \qquad \forall i \neq 1$ \\\\\\
Each Pointer ends when the previous MDS does. \\\\\\
$(8) \qquad P_{end}(i,j) = MDS_{MACEnd}(i-1,j) \qquad \forall i \neq 1$ \\\\\\
MAC Pointer Coverage \\\\\\
$(9) \qquad Cov_{MACPOINT}(i,j) = \mathlarger{\sum_{b \geq j} P_{start}(i,b) + \sum_{e < j} P_{end}(i,e) \qquad \forall i}$ \\\\\\
100\% MAC coverage, every part should be covered by (at least) one MDS. \\\\\\
$(10) \qquad \mathlarger{\sum_{i} Cov_{MAC}(i,j) \geq 1 \qquad \forall j }$ \\\\\\
\clearpage
Overlap sections are covered by 2 MDS \\\\
$(11) \qquad \mathlarger{\sum_{i} Cov_{MAC}(i,j) \leq 2 \qquad \forall j }$ \\\\\\

% FIXME: <=, > not <= =>.

\section{Preprocessing}
This part of the software computes the value for some of the variables, taking the instance as input.

Some of the defined variables are \textit{4-dimensional MIC length $\times$ MIC length $\times$ MAC length $\times$ MAC length} arrays. The necessity of a sparse data structure was immediatly clear: \textit{Sparray}, a Python module \cite{sparray} for sparse n-dimensional arrays using \textit{dictionaries} supporting any number of dimensions and any size was chosen to support these variables.

The Read-Write performance on these objects is satisfying: 15 milion integer values are written in random indexes in a 150M $\times$ 150M $\times$ 150M $\times$ 150M \textit{4d sparray} in less than 20 seconds. The data can then be accessed using a notation similar to the standard array one (\texttt{Sparray[Index1, Index2, Index3, Index4]}). \textit{Eq} and {cwc} are populated during this phase, with a naive iterative algorithm.

Here we encounter our first big limitation of a \textit{pure} linear programming approach. Populating Eq and cwc in our test instance (60 characters long MIC) was trivial but this task becomes so expensive it's infeasible with even genome sequences of more than 1000 characters.

The subsequences matching part must be approached with an high performing alignment tool like \textit{BLAST}.

\section{Gurobi Implementation}
gurobi/py general structure

gurobi constraint examples

TODO

Here's how some of the constraints described are translated:

TODO

\section{Correctness}
\label{correctness}

\paragraph{Lemma.}
Suppose

$I$ to be an instance of the problem,

$P$ to be the ILP formulation associated to $I$,

$S$ to be an assignment to every variable of $P$ satisfying the constraints.

Then it's possible to build a solution for $I$ with cost equal to the objective function in $S$.

\paragraph{Proof.}
The idea is checking if \textit{S} produced by Gurobi is compatible with the known rearrangement map of the instance, i.e., if the solution proposed by Gurobi represents our DNA recombination events. "Applying" the computed map to the instance MIC should build an exact copy of the MAC.

The complete test workflow follows.

\begin{enumerate}
	\item Produce an \textit{artificial} instance \textit{I} of the problem with the software described in the \textit{Reduced artificial instance} section. By definition, this instance exhibits the rearrangement events we want to study (Scrambling, Overlapping, Deletion, Inversion) and it's a compliant instance of the formalised problem. A \textit{known} rearrangement map $R_1$ is produced too.
	\item Run the preprocess script on the generated instance and populate some of the variables of the ILP formulation associated to \textit{I}
	\item Run the Gurobi implementation of $P$ passing the populated variables. An assignment $S$ of every variable of the formulation will be computed.
	\item Build a rearrangement map $R_2$ using $S$.
	\item If $S$ is a solution then it $R_2$ maps the rearrangement events. It is possibile to simulate those events on the instance MIC and exactly obtain the MAC. $R_2$ is comparable with $R_1$.
\end{enumerate}

Step 1 is described in \ref{gen} while steps 2-4 are handled in the \texttt{ilp.py} script. Step 5 is covered in \texttt{map.py} where we provide a function that computes a final rearranged sequence given an initial sequence and a (compliant) rearrangement map \ref{rmap}.

This entire procedure can be pipelined and automatized to allow running a variety of tests.

\section{Conclusions}

A \textit{pure} integer linear programming approach in this terms is clearly not successful, given the magnitude of the data to process and our choices.

A mixed approach should give better results: preprocessing the common substrings (BLAST) and producing possible instances consisting of compatible subsets of matching substrings speeds up the initial part and won't bloat the ILP formulation with huge matrices of data.

A scoring function must be designed to measure the quality of the possible maps: an ILP formulation could then optimise the problem.

Note that we (and other approaches like MIDAS \cite{midas}) use a greedy criteria for MDSs annotation: a solution with the largest possibile MDSs annotation is considered the best, requiring a 100\% MAC coverage.

As far as we know the process could actually behave differently: the availability of transitional genomes, showing the process during intermediate phases, could change this view.